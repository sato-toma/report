\documentclass[a4paper,11pt]{jsbook}


% 数式
\usepackage{amsmath,amsfonts}
\usepackage{bm}
% 画像
\usepackage[dvipdfmx]{graphicx}


\begin{document}

\title{初めてのデッサン日記}
\author{さとうとーま}
\date{\today}
\maketitle
\tableofcontents

\part{初めに}
\chapter{何を記述して何を記述しないか}
\chapter{準備}
\section{道具の準備}
\section{心の準備}
\chapter{鉛筆の使い方}
\chapter{グラデーション}
\chapter{消しゴムの使い方}

\part{幾何デッサン}
\chapter{立方体}
\section{輪郭を描く}
\section{濃淡を描く}
\section{表面を描く}

\chapter{円柱}
\section{輪郭を描く}
\section{濃淡を描く}
\section{表面を描く}

\chapter{円錐}
\section{輪郭を描く}
\section{濃淡を描く}
\section{表面を描く}

\chapter{円錐}
\section{輪郭を描く}
\section{濃淡を描く}
\section{表面を描く}

\chapter{円錐角柱相貫体}
\section{輪郭を描く}
\section{濃淡を描く}
\section{表面を描く}

\chapter{角柱角柱相貫体}
\section{輪郭を描く}
\section{濃淡を描く}
\section{表面を描く}

\chapter{正十二面体}
\section{輪郭を描く}
\section{濃淡を描く}
\section{表面を描く}

\chapter{球体}
\section{輪郭を描く}
\section{濃淡を描く}
\section{表面を描く}

\part{複数モチーフデッサン}

\part{石膏人物デッサン}

\part{動物デッサン}

\part{終わりに}
\chapter{謝辞}

\appendix
\chapter{A}




\thispagestyle{empty}
\vspace*{\stretch{1}}
\begin{flushright}
\begin{minipage}{0.5\hsize}
\begin{description}
  \item{著者:}さとうとーま
  \item{挿絵:}さとうとーま
  \item{発行:}\date{\today}
  % \item{印刷:}POPLS (\verb|http://www.inv.co.jp/~popls/|)
\end{description}
\end{minipage}
\end{flushright}

\end{document}


